\chapter{Conclusion}
We presented a class of coarsening methods for finite element simulation of elastic materials.
These methods speed up the simulation of non-linear elastic materials to be used in iterative design algorithms while still predictive enough to improve the design.
The accuracy and efficiency of our methods are demonstrated by designing and fabricating functional 3D prints with specified static and dynamic deformation properties.

Traditionally, engineers improve the accuracy of FEM simulation by adaptively refining the mesh until some convergence test is satisfied.
The elements must be small enough to capture the geometric and material variations.
Moreover, due to numerical stiffening, the element sizes must be a fraction of the feature sizes.
Because of such requirements, accurate simulation of a detailed design can take from minutes to hours on a desktop computer with a single CPU.
Homogenization and numerical coarsening methods try to overcome the efficiency problem by computing new material models for the coarse elements.
Our coarsening approach generalizes the previous methods to handle general non-linear elastic materials and for dynamic scenarios. In theory, any basis function can be used to fit the behavior of coarse elements. We performed experiments with two basis functions:the Neo-Hookean constitutive model and a spring-like energy function for modeling direction of anisotropy.

