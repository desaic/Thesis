\section{Inverse Design for Elastostatics}
Many engineering problems focus on the design of complex structures that needs to meet high level objectives such as the capability to support localized stresses, optimal tradeoffs between compliance and mass, minimal deformation under thermal changes, etc. One very popular approach to design such structures is topology optimization. It computes material distribution over elements to optimize given functional goals \cite{bendsoe2004topology}. Traditionally, it focused on designs made of homogeneous materials and was concerned with macroscopic changes in the object geometry.
With the advent of multi-material 3D printing techniques, it is now possible to manufacture objects at a much higher resolution, allowing much finer designs and improved functional performances.
Unfortunately, standard techniques for topology optimization do not scale well and they cannot be run on objects with billions of voxels. This is because the number of variables to optimize increases linearly with the number of cells. Since many current 3D printers have a resolution of 600DPI or more, a one billion voxel design occupies only a 1.67 inch cube.

One direction to handle this issue is to work with microstructures corresponding to blocks of voxels instead of individual voxels directly. Some recent works followed this direction and proposed to decouple macro structural design and micro material design \cite{rodrigues:2002:hierarchical,coelho:2008:hierarchical,nakshatrala:2013:nonlinear}. However, these approaches remain computationally expensive and, in most cases, limited to the well-known minimal compliance problem.
Closer to our present work, frameworks for the design of objects with desired mechanical behaviors have been proposed by Bickel et al. \cite{Bickel2010} and Skouras et al. \cite{Skouras2013}. Like these works, our system  allows to match given input deformations. However, while these previous systems assume a small set of available base materials and use these base materials in relatively coarse discretizations, our system combines the base materials into microstructures to expand the design possibilities. Also relevant is the tool presented by Xu et al. \cite{Xu2014} that allows to interactively design heterogeneous materials for elastic objects subject to prescribed displacements and forces, and the material optimization approach proposed by Panetta et al. \cite{Panetta:2015}. However, these methods may output materials that are not available in the real world for non-convex manifolds of material properties. By contrast, we guarantee that all the microstructures used are always realizable in such cases, which is one of the key contributions of our work. 